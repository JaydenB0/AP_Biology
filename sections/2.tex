\documentclass[../main.tex]{subfiles}
\begin{document}
\begin{outline}
	\1 Organisms are composed of matter
	\2 Matter --- anything that takes up space and as mass
	\3 Mass --- the amount of matter in an object
	\3 Weight --- how strongly the mass is pulled by gravity
	\1 Element --- substance that cannot be broken down to other substances by chemical reactions
	\1 Compound --- substance consisting of two or more different elements in a fixed ratio
	\1 25 of 92 elements are known to be essential to life
	\2 Carbon, hydrogen, oxygen, and nitrogen (CHON) make up 96 percent of living matter
	\2 Trace elements are required only in minute quantites
	\1 Atom --- Smallest unit of matter that still retains property of element
	\1 Dalton = AMU
	\1 Atomic number is the amount of protons; goes on the bottom
	\1 Mass number = sum of protons plus neutrons; goes on top
	\2 Remember from Chemistry, a mole of an element is its mass number in grams; 6.022x10$^{23}$
	\1 Atomic mass is an average
	\1 Isotopes are different atomic forms of an element
	\2 \hl{Isotopes contain amounts of NEUTRONS}
	\1 Radioactive isotope decays spontaneously
	\1 Energy --- capacity to cause change
	\1 Potential energy --- energy that matter possess because its location or structure
	\2 Energy has a tendency to move to the lowest state of potential energy
	\2 It takes work to move an electron further away from the nucleus
	\2 The farther electrons are from nucleus, the greater the potential energy
	\1 Electron's energy levels are correlated with average distance from nucleus
	\2 Average distances are electron shells
	\2 When electron absorbs energy, moves to a farther shell
	\1 Lost energy is usually release in a form of heat
	\1 Valence electrons = outermost electrons, outermost electron shell = valence shell
	\1 3 dimensional space where an electron is found 90\% of the time is an orbital
	\1 First electron shell has only 1 spherical S orbital (called 1s)
	\1 Second electron shell has 4 orbitals
	\2 1 large S orbital called 2s
	\2 3 dumbbell shaped p orbitals (called 2p orbitals)
	\2 Each 2p orbital is oriented at right angles to the other 2p orbitals
	\1 No more than 2 electrons can occupy a single orbital
	\1 Covalent bonding is sharing a pair of valence by 2 atoms
	\1 \hl{Two or more atoms held by a covalent bond make a molecule}
	\2 Only covalent bonds make molecules
	\1 H---H: This notation represents a \textbf{STRUCTURAL FORMULA}
	\1 H$_2$ represents a \textbf{MOLECULAR FORMULA}
	\1 Valence --- bonding capacity of an atom
	\2 usually equals the number of unpaired electrons in the atom's valence shell
	\3 Phosphorus has a valence of 5
	\1 Electronegativity --- the attraction of an atom for the electrons of a covalent bond
	\2 The more electronegative, the more strongly it pulls shared electrons
	\2 2 atoms of the same element are equally electronegative
	\1 Nonpolar covalent bond --- electrons shared equally
	\2 Polar Covalent bond --- bonds not shared equally
	\3 O is very electronegative, in water, the electrons spend more time near oxygen so oxygen becomes negative
	\1 When 2 atoms are very unequal in attraction, ionic bond forms
	\1 ion --- charged atom
	\1 cation --- positively charged ion
	\1 anion --- negatively charged ion
	\1 Any ions of opposite charges can form an ionic bond
	\1 Ionic compounds or salts --- compounds formed by ionic bonds
	\1 hydrogen bond --- weak chemical bond
	\2 forms when a hydrogen atom covalently bonded to one electronegative atom is also attracted to another
	\1 Hybridization --- single s and 3 p oribtals hybridize to form 4 teardrop shaped orbitals
	\1 chemical equilibrium --- point at which reactions offset one another exactly, no net chage in reactants and products
\end{outline}
\end{document}
