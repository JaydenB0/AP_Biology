\documentclass[../main.tex]{subfiles}
\begin{document}
\begin{outline}
	\1 Cells themselves are about 70-95\% water
	\1 Water is 2 hydrogen atoms joined to oxygen by singles covalent
	\1 Water is a polar covalent molecule
	\1 Cohesion
	\2 Hydrogen bonds are 1/20th as strong as covalent
	\3 Only last a few trillionths of a second
	\2 Hydrogen bonds hold the substance together
	\2 \hl{contributes to transport of water and nutrients against gravity in plants}
	\1 Adhesion
	\2 Hydrogen bonds cause water molecules leaving to tug on molecules fartherdown
	\2 Helps counter the downward pull of gravity
	\1 Surface tension --- meausre of how difficult it is to stretch or break the surface of a liquid
	\2 Water has a greater surface tension than most liquids
	\1 Absorbs heat from warmer air and releases heat to air thats colder
	\2 slightly change its own temperature
	\1 Heat --- measure of the total amount of kinetic energy due to molecular motion
	\1 Temperature --- intensity of heat due to average kinetic energy of molecules
	\1 Calorie --- amount of heat it akes to raise the temperature of 1g of water by 1C
	\2 also amount of heat that 1g releases when it cools by 1C
	\1 Kilocalorie, 1000 cal, is quantity of heat required to heat 1kg of water by 1C
	\1 Joule = 0.239cal, Cal = 4.184J
	\1 Specific heat --- amount of heat that must be absorbed or lost for 1g to change its temp by 1C
	\2 Specific heat of water = 1/cal/g/C
	\1 Water has high specifc heat bc most of heat is used to disrupt bonds before the water moves faster
	\2 When water temp drops, more hydrogen bonds form
	\1 Heat of vaporization --- quantity of heat a liquid must absorb for 1g to convert into gas
	\2 1g of water at 25C takes 580cal of heat to evaporate
	\1 As liquid evaporates, surface cools down because molecules with most kinetic likely leave as gas
	\1 Water expands when solidify
	\2 caused by hydrogen bonds
	\2 when frozen, molecules dont break their bonds
	\2 stuck in crystalline lattice
	\3 bonds partnered to 4 partners
	\2 causes ice to become 10\% less dense
	\1 Lowered density of ice causes it to float on water
	\2 important because if ice sank, oceans would freeze solid
	\2 insulates water below
	\1 solution --- homogeneous mixture of two or more substances
	\1 solvent --- dissolving agents of a solution
	\1 solute --- substance being dissolved
	\1 aqueous solution --- solution with water solvent
	\1 mythical universal solvent --- something that would dissolve anything
	\2 would dissolve the container
	\1 Water solvent quality traced to \hl{polarity of water molecule}
	\2 Salt placed in water causes oxygen to separate sodium and hydrogen to the chloride
	\1 sphere of water molecules around dissolved ion is called a hydration shell
	\1 compounds made of polar covalent bonds can dissolve in water
	\2 compounds dissolve when water molecules surround each of the solute molecules
	\1 hydrophillic --- substance with affinity for water
	\2 some substances can by hydrophillic without dissolving bc large suize
	\3 they remain suspended in aqueous liquid of the cell
	\1 colloid --- stable suspension of fine particles in a liquid
	\1 cotton doesn't dissolve in water because large molecules of cellulose --- compound with positive and negative polar bonds
	\2 cellulose is also in walls of plant
	\1 hydrophobic --- water repelling
	\2 nonionic and nonpolar molecules repel water
	\1 molecular mass --- sum of masses of all atoms in molecule
	\1 molarity --- number of moles of solute per liter of solution
	\2 unit of concentration most used by biologists for aqueous solutions

\end{outline}
\end{document}
